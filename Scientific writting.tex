\documentclass{article}
\title{Scientific writing demonstration}
\author{Qi Wang}
\begin{document}
    \maketitle
    \section*{introduction}
    First two sentense to introduce background of the topic as well as its lack of knowledge, in understandable words, not complete and long clauses.Explain the importance of the lacking knowledge. How this propses method will help to fullfill or improve the situation. Some general description of method could be added here. What the results tell the reader, regarding the afore-mentioned problem.
    \section*{method}
    Usually, separate data acquisition part with method part into paragraphs.
    Describe how data was acquired and processed. 
    Describe how the method is designed.
    \section*{results}
    Objectively, describe what was shown in figrues. 
    \section*{conclusion}
    As a response to introduction, talk about what question was solved in the work, and how it was done. A comment on the work and how it could be extended for future works.
    \section*{discussion}
    Talk about the shortage and problems encountered in the work. 
\end{document}